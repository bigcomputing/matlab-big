\section{Data Sharing Tutorial}
NetWorkSpaces (NWS) is an MATLAB package that makes it very easy to
communicate and coordinate multiple MATLAB programs running on different
machines.  In this chapter, we will give a small walk-through of writing
a simple program using NetWorkSpaces.  Appendix A
contains a detailed reference of the NetWorkSpace class.

First, start an interactive MATLAB session.

Next, create a workspace called `bayport':

\begin{verbatim}
>> ws = netWorkSpace('bayport');
\end{verbatim}

This workspace is created on a NWS server located on localhost, port
8765.  Additional arguments can be passed to the netWorkSpace
constructor to specify a NWS server located on a different host and
port.

\begin{samepage}
\begin{verbatim}
>> opt.host = 'mercury';
>> ws2 = netWorkSpace('bayport', opt);
\end{verbatim}
\end{samepage}

Once we have created a workspace, we can write data into a variable
using the store method:

\begin{verbatim}
>> store(ws, 'joe', 17);
\end{verbatim}

The variable `joe' now has a value of 17 in the workspace named
`bayport'.

To find out the value associates with the variable `joe', we use the
find method:

\begin{verbatim}
>> age = find(ws, 'joe');
\end{verbatim}

which sets `age' to 17.

Note that the find method will block until the variable `joe' has a
value.  This is important when we're trying to read that variable from a
different machine.  If it didn't block, you might have to repeatedly try
to read the variable until it succeeded.  Of course, there are times
when you don't want to block, but just want to see if some variable has
a value.  That is done using the findTry method.

\begin{verbatim}
>> age = findTry(ws, 'chet');
\end{verbatim}

This assigns `0' to age, since we didn't stored any value to variable
`chet'.  If you'd rather have findTry return some other value when the
variable doesn't exist, you can pass in an extra argument to the findTry
method:

\begin{verbatim}
>> age = findTry(ws, 'chet', 1);
\end{verbatim}

which assigns 1 to age, if variable `chet' doesn't exist.

So far, we've carefully executed the store method once for each variable
created.  What if we execute the store method more than once? Then, it
depends on the mode of the variable. By default, a variable created by
the store method uses first-in-first-out (FIFO), which means the first
stored value in the workpace will be the first one to be read. If you
want to use a different mode for the variable, you have to use the
declare method to create the variable first.  Please refer to
Appendix A for a description of the different variable modes.

Since the find method does not remove the value from the workspace, we cannot
use the find method to read multiple values associate with a variable. To
read through all the values, we have to use `fetch' method.  The fetch
method works the same as find, but in addition, it removes the value.
This can be useful in sending a sequence of messages from one program to
another.

Let's try writing multiple values to a variable:

\begin{samepage}
\begin{verbatim}
>> n = [16 19 25 22]
>> for i = 1:length(n)
     store(ws, 'biff', n(i))
   end
\end{verbatim}
\end{samepage}

To read the values, we just call fetch repeatedly:

\begin{samepage}
\begin{verbatim}
>> n = []
>> for i = 1:4
     n = [n fetch(ws, 'biff'))
   end
\end{verbatim}
\end{samepage}

If we didn't know how many values were stored in a variable, we could
have done the following:

\begin{samepage}
\begin{verbatim}
>> n = []
>> while 1
     t = fetchTry(ws, 'biff');
     if t == 0 break; end;
     n = [n t];
   end
\end{verbatim}
\end{samepage}

This uses fetchTry, which works like fetch, except that it is
non-blocking.

These are the basic operations provided by NWS.  It's a good idea to
play around with these operations using two MATLAB sessions.  That way,
you can really transfer data between two different programs.  Also, you
can see how the blocking operations work.  In the examples above, we
were careful never to block because we were only using one MATLAB
session.

To use two MATLAB sessions, just execute MATLAB in another window,
export MATLABPATH or have it saved and load up automatically when a new
window is opened.  Then, open the `bayport' workspace.  This can be done
with the same command that we used previously, but this time, the
command:

\begin{verbatim}
>> ws = netWorkSpace('bayport')
\end{verbatim}

won't create the `bayport' workspace, since it already exists.  Now you
can execute a operation such as:

\begin{verbatim}
>> x = fetch(ws, 'frank')
\end{verbatim}

in one session, watch it block for a minute, and then execute store in
the other session:

\begin{verbatim}
>> store(ws, 'frank', 18)
\end{verbatim}

and see that the fetch in the first session completes.

While you're experimenting with these operations, it can be very helpful
to use the NWS server's web interface to see what's going on in your
workspaces. Just point a web browser to: \url{http://localhost:8766}

If you're using a browser on another machine from the NWS server you'll
have to use the appropriate hostname, rather than `localhost'.
