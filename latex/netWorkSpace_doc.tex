\section {netWorkSpace Class}
%HEVEA\aname{netWorkSpace}{
\rule[0.06in]{6in}{0.01in}
\newline
netWorkSpace
\newline
\rule{6in}{0.01in}
\begin{list}{}{}
	\item {\bf Description\\}
	netWorkSpace class constructor.
	\item {\bf Usage\\}
	ws = netWorkSpace(wsName, h, p);
	or \\
	ws = netWorkSpace(wsName, opts);
	\item {\bf Arguments}
 		\begin{tabbing}
		wsName \hspace{2.5cm} \= name of the workspace.\\
		h \> server host name.\\
		p \> server port.\\
		opts \> an optional structure of fields. see details.\\
		\end{tabbing}
	\item {\bf Details\\}
	netWorkSpace(ws, h, p) or netWorkSpace(ws, opts) creates an object for the shared netWorkSpace wsName residing in the netWorkSpaces server 
	running on host h, port p. If host name and port number are not provided, then `localhost' is used as default
	value for host name, and 8765 is used as default value for port.

	opts is an optional structure of fields. If opts is presented, then host name and port number is specified through opts instead
	of arguments to the function.

	opts may contain these fields:
	\begin{itemize}
		\item \textit {server}: nwsServer class object. This field is not needed, if host name and port number are provided. 
		\item \textit {host}: server host name. Default value is `localhost'.
		\item \textit {port}: server port. Default value is 8765. 
		\item \textit {persistent}: If true, make the workspace persistent. A persistent workspace won't be purged when the owner
		disconnects from an nws server. Note that only the client who actually takes ownership of the workspace can make 
		the workspace persistent. The persistent argument is effectively ignored if useUse is true. 
		\item \textit {useUse}: By default `open' semantic is used when creating a workspace (i.e., trying to claim ownership). 
		If useUse is set to 1, `use' semantic applies, which means no ownership is claimed.\\
	\end{itemize}
	
	\item {\bf Examples}
		\begin{verbatim}
		example 1: create a workspace with default options.
		>> nws = netWorkSpace(`nws example')
	
		example 2: connect to a workspace using useUse semantic.
		>> opt.useUse = 1
		>> nws = netWorkSpace(`example2', opt)
	
		example 3: create a workspace on an nws server that
				   resides on node1, port 5555.
		>> nws = netWorkSpace(`example3', `node1', 5555)
		\end{verbatim}
\end{list}
%HEVEA }

%HEVEA\aname{fetch}{
\rule[0.06in]{6in}{0.01in}
\newline
fetch
\newline
\rule{6in}{0.01in}
\begin{list}{}{}
	\item {\bf Description}\\
	Fetch a value associates with the variable from the shared netWorkSpace nws.
	\item {\bf Usage}\\
	X = fetch(nws, varName);
	\item {\bf Arguments}
		\begin{tabbing}
		nws	\hspace{2.5cm} \= a netWorkSpace class object.\\
		varName	\> name of the variable to fetch.\\
		\end{tabbing}
	\item {\bf Details}\\
	Fetch blocks until a value for varName is found in the shared netWorkSpace nws.
	Once found, remove a value associated with varName from the shared netWorkSpace and return it in X. 
	This operation is atomic. If multiple Distributed Computing Toolbox sessions fetch or fetchTry a 
	given varName, any given value from the set of values associated with varName will be return to 
	just one DCT session. If there is more than one value associated with varName, the particular value 
	removed depends on varName's behavior. See declare(nws, ...) for details.
	
	\item {\bf Examples}
		\begin{verbatim}
		>> nws = netWorkSpace(`nws example');
		>> store(nws, `x', 1);  % store variable x with value 1 to nws
		>> X = fetch(nws, `x'); % X = 1
		>> X = fetch(nws, `x'); % no value for x; block on fetch
		\end{verbatim}
\end{list}
%HEVEA}

%HEVEA\aname{fetchTry}{
\rule[0.06in]{6in}{0.01in}
\newline
fetchTry
\newline
\rule{6in}{0.01in}
\begin{list}{}{}
	\item {\bf Description}\\
	Attempt to fetch a value associates with the variable from the shared netWorkSpace nws; 
	a non-blocking version of fetch(nws, ...).
	\item {\bf Usage\\}
	[X, F] = fetchTry(nws, varName, ifMissing);
	\item {\bf Arguments}
		\begin{tabbing}
		nws	\hspace{2.5cm} \= a netWorkSpace class object. \\
		varName	\> name of variable to fetch. \\
		ifMissing \> value to return, if variable by varName is not found in netWorkSpace nws.
		\end{tabbing}
	\item {\bf Details}\\
	Look in the shared netWorkSpace nws for a value bound to varName.
 
	If found, remove a value associated with varName from nws and
	return it in X. Set F to true. This operation is atomic. If
	multiple DCT sessions fetch or fetchTry a given varName, any
	given value from the set of values associated with varName will
	be return to just one DCT session. If there is more than one
	value associated with varName, the particular value removed
	depends on varName's behavior. See declare(nws, ...) for details.
    
	If not found, return the value of ifMissing in X and set F to false.
	Default value for ifMissing is zero. 

	\item {\bf Examples}
		\begin{verbatim}
		>> nws = netWorkSpace(`nws example');
		>> store(nws, `x', 1);
		>> 
		>> [X, F] = fetchTry(nws, `x'); % X = 1, F = 1
		>> [X, F] = fetchTry(nws, `x'); % X = 0, F = 0 
		\end{verbatim}
\end{list}
%HEVEA}

%HEVEA\aname{find}{
\rule[0.06in]{6in}{0.01in}
\newline
find
\newline
\rule{6in}{0.01in}
\begin{list}{}{}
	\item {\bf Description}\\
	Find a value associates with the variable in the shared netWorkSpace nws.
	\item {\bf Usage}\\
	X = find(nws, varName);
	\item {\bf Arguments}
		\begin{tabbing}
		nws	\hspace{2.5cm} \= a netWorkSpace class object. \\
		varName \> name of the variable to find.
		\end{tabbing}
	\item {\bf Details}\\
	find(nws, varName) blocks until a value for varName is found in the shared netWorkSpace nws.
    
	Once found, return in X a value associated with varName, but the value is not removed from
	netWorkSpce, nws. If there is more than one value associated with varName, 
	the particular value returned depends on varName's behavior. See declare(nws, ...) for details.
	\item {\bf Examples}
		\begin{verbatim}
		>> nws = netWorkSpace(`nws example');
		>> store(nws, `x', 1);
		>> X = find(nws, `x');   % X = 1
		>> listVars(nws); % X value is not removed from nws
		\end{verbatim}
\end{list}
%HEVEA }


%HEVEA\aname{findTry}{
\rule[0.06in]{6in}{0.01in}
\newline
findTry
\newline
\rule{6in}{0.01in}
\begin{list}{}{}
	\item {\bf Description}\\
	Attempt to find a value associates with the variable in the shared netWorkSpace nws;	a non-blocking version of find(nws, ...).
	\item {\bf Usage\\}
	[X, F] = findTry(nws, varName, isMissing);
	\item {\bf Arguments}
		\begin{tabbing}
		nws \hspace{2.5cm} \=  a netWorkSpace class object.\\
		varName \> name  of the variable to find.\\
		isMissing \> value to return if no variable by varName is found in netWorkSpace nws.
		\end{tabbing}
	\item {\bf Details}\\
	$[$X, F$]$ = findTry(nws, varName, isMissing) looks in the nws for a value bound to varName.
      
	If found, returns a value associated with varName in X and sets F to true. 
	No value is removed from netWorkSpace, nws. If there is more than one value associated with 
	varName, the particular value returned depends on varName's behavior. 
	See declare(nws, ...) for details.
      
	If not found, return the value of isMissing in X and set F to false.
	Default value for ifMissing is zero. 

	\item {\bf Examples}
		\begin{verbatim}
		>> nws = netWorkSpace(`nws example');
		>> [X, F] = findTry(nws, `x'); % check if variable x exists
		\end{verbatim}
\end{list}
%HEVEA }

%HEVEA\aname{store}{
\rule[0.06in]{6in}{0.01in}
\newline
store
\newline
\rule{6in}{0.01in}
\begin{list}{}{}
	\item {\bf Description}\\
	Store a value associates with the variable in the netWorkSpace, nws.
	\item {\bf Usage}\\
	store(nws, varName, value);
	\item {\bf Arguments}
		\begin{tabbing}
		nws \hspace{2.5cm} \= a netWorkSpace class object.\\
		varName \> variable name.\\
		value \> value associates with varName to be stored in nws. 
		\end{tabbing}
	\item {\bf Details}\\
	store(nws, varName, v) associates the value v with the
	variable varName in the shared netWorkSpace nws, thereby making
	the value available to all the instances of MATLAB running in
	the current Distributed Computing Toolbox session.  value may be
	any MATLAB value -- matrix, structure, function handle, etc.

	If a mode has not already been set for varName, `fifo' will be used (see declare(nws, ...)).

	If no variable name is given, then value v must itself be a variable,
	and the value v will be asssociated with the variable's name in
	the netWorkSpace nws. See the examples below for details.

	Note that, by default (`fifo' mode), store is not idempotent:
	repeating store(nws, `X', v) will add additional values to the
	set of values associated with the name `X'. See the examples below for details.

	\item {\bf Examples}
		\begin{verbatim}
		>> M = magic(5); 
		>> store(nws, M);
		\end{verbatim}
		and
		\begin{verbatim}
		>> store(nws, `M', magic(5));
		\end{verbatim}
		do the same thing. 
	
		\begin{verbatim}
		>> for x = 1:10
		store(nws, `numList', x);
		end
		\end{verbatim}
	
		will associated ten distinct values with the name `numList'.
	
		\begin{verbatim}
		>> for i = 1:10
		fetch(nws, `numList');
		end
		\end{verbatim}
	
		will retrieve these distinct values, one at a time from the
		first to last stored.
\end{list}
%HEVEA }

%HEVEA\aname{declare}{
\rule[0.06in]{6in}{0.01in}
\newline
declare
\newline
\rule{6in}{0.01in}
\begin{list}{}{}
	\item {\bf Description}\\
	Declare a variable with a particular mode in the shared netWorkSpace.
	\item {\bf Usage}\\
	X = declare(nws, varName, mode);
	\item {\bf Arguments}
		\begin{tabbing}
		nws	\hspace{2.5cm} \= a netWorkSpace class object.\\
		varName \> variable name.\\
		mode \>	variable mode.
		\end{tabbing}
	\item {\bf Details}\\
	If varName has not already been declared in nws, the behavior
	of varName will be determined by mode. Mode can be `fifo',
	`lifo', `multi', or `single'. In the first three cases,
	multiple values can be associated with varName. When a value is
	retrieved for varName, the oldest value stored will be used in
	`fifo' mode, the youngest in `lifo' mode, and a
	nondeterministic choice will be made in `multi' mode. In
	`single' mode, only the most recent value is retained.
	\item {\bf Examples}
		\begin{verbatim}
		>> nws = netWorkSpace(`nws example');
		>> declare(nws, `pi', `single');
		>> store(nws, `pi', 2.171828182);
		>> store(nws, `pi', 3.141592654);
		\end{verbatim}

	listVars(nws) will show that only the most recent value of pi is retained.
\end{list}
%HEVEA }

%HEVEA\aname{deleteVar}{
\rule[0.06in]{6in}{0.01in}
\newline
deleteVar
\newline
\rule{6in}{0.01in}
\begin{list}{}{}
	\item {\bf Description}\\
	Delete a variable from the shared netWorkSpace.
	\item {\bf Usage}\\
	X = deleteVar(nws, varName);
	\item {\bf Arguments}
		\begin{tabbing}
		nws \hspace{2.5cm} \= a netWorkSpace class object.\\
		varName \> variable name of the object to delete.
		\end{tabbing}
	\item {\bf Examples}
		\begin{verbatim}
		>> nws = netWorkSpace(`nws example');
		>> store(nws, `x', 1);
		>> deleteVar(nws, `x');
		>> listVars(nws); % no variables in nws example 
		\end{verbatim}
\end{list}
%HEVEA }

%HEVEA\aname{listVars}{
\rule[0.06in]{6in}{0.01in}
\newline
listVars
\newline
\rule{6in}{0.01in}
\begin{list}{}{}
	\item {\bf Description}\\
	List variables in the netWorkSpace, nws.
	\item {\bf Usage}\\
	X = listVars(nws, netWorkSpaceName);
	\item {\bf Arguments}
		\begin{tabbing}
		nws \hspace{3.0cm} \= a netWorkSpace class object.\\
		netWorkSpaceName \> netWorkSpace name.
		\end{tabbing}
	\item {\bf Details}\\
	X = listVars(netWorkSpaceName) returns a struct array listing
	of variables in the named netWorkSpace. The listing contains number of values,
	fetchers, finders and mode for each variable. If netWorkSpaceName argument
        is not provided, then list variables in the netWorkSpace reference by the first argument, nws. 
     
	If no left hand side, print a tabular listing.
	\item {\bf Examples}
		\begin{verbatim}
		>> nws = netWorkSpace(`nws example');
		>> store(nws, `x', 1);
		>> listVars(nws);
		\end{verbatim}
\end{list}
%HEVEA }

%HEVEA\aname{currentWs}{
\rule[0.06in]{6in}{0.01in}
\newline
currentWs
\newline
\rule{6in}{0.01in}
\begin{list}{}{}
	\item {\bf Description}\\
	Report the name of a shared netWorkSpace.
	\item {\bf Usage}\\
	X = currentWs(nws);
	\item {\bf Arguments}
		\begin{tabbing}
		nws	\hspace{2.5cm} \= a netWorkSpace class object.
		\end{tabbing}
	\item {\bf Examples}
		\begin{verbatim}
		>> nws = netWorkSpace(`nws example');
		>> wsName = currentWs(nws);
		>> display(wsName); % wsName = nws example 
		\end{verbatim}
\end{list}
%HEVEA }

%HEVEA\aname{close}{
\rule[0.06in]{6in}{0.01in}
\newline
close
\newline
\rule{6in}{0.01in}
\begin{list}{}{}
	\item {\bf Description}\\
	Close connection of a shared netWorkSpace to the NetWorkSpaces server. 
	\item {\bf Usage}\\
	X = close(nws);
	\item {\bf Arguments}
		\begin{tabbing}
		nws \hspace{2.5cm} \= a netWorkSpace class object.
		\end{tabbing}
	\item {\bf Examples}
		\begin{verbatim}
		>> nws = netWorkSpace(`nws example');
		>> close(nws);
		\end{verbatim}
\end{list}
%HEVEA }

%HEVEA\aname{server}{
\rule[0.06in]{6in}{0.01in}
\newline
server
\newline
\rule{6in}{0.01in}
\begin{list}{}{}
	\item {\bf Description}\\
	Return an instance of the NetWorkSpaces server that the netWorkSpace, nws, connects to.
	\item {\bf Usage}\\
	s = server(nws);
	\item {\bf Arguments}
		\begin{tabbing}
		nws \hspace{2.5cm} \= a netWorkSpace class object.
		\end{tabbing}
	\item {\bf Examples}
		\begin{verbatim}
		>> s = server(nws);
		\end{verbatim}
\end{list}
%HEVEA }

