\section{Sleigh for MATLAB Tutorial}
Sleigh is a MATLAB class, built on top of NWS, that makes it very easy to
write simple parallel programs.  It provides two basic functions for
executing tasks in parallel: \textit{eachElem} and \textit{eachWorker}.

\begin{description}
\item[eachElem] is used to execute a specified function multiple times in
parallel with a varying set of arguments.
\item[eachWorker] is used to execute a function exactly once on every worker in
the sleigh with a fixed set of arguments.
\end{description}

\texttt{eachElem} is all that is needed for most basic programs, so that
is what we will start with.

First, we need to start up a sleigh.

\begin{verbatim}
>> s = sleigh
\end{verbatim}

This starts three sleigh workers on the local machine, but the number of
sleigh workers can be changed by specifying workerCount field of the
struct argument. 

Let's shut down sleigh so we can start different number of workers on
local machine. 

\begin{verbatim}
>> stop(s)
\end{verbatim}

This deletes the sleigh's NWS workspace, and shuts down all of the
sleigh worker processes.

Now we'll make a new sleigh, with two workers on local machine, and
we'll use an NWS server that's running on node10:

\begin{samepage}
\begin{verbatim}
>> opt.nwsHost = 'node10';
>> opt.workerCount = 2;
>> s = sleigh(opt)
\end{verbatim}
\end{samepage}

Now we're ready to run a parallel program.  Here it is:

\begin{verbatim}
>> result = eachElem(s, 'add1', 1:10);
\end{verbatim}

In this simple command, we have defined a set of data that is processed
by multiple workers in parallel, and returned each of the results in a
cell array.  Of course, you would never really bother to do such a
trivial amount of work with a parallel program, but you get the idea.

The second argument is the name of the function workers will evaluate.
This assumes that workers have an M function called add1 somewhere in
the system.  If not, you can create a file called `add1.m' with the
content as follows:

\begin{samepage}
\begin{verbatim}
function y = add1(x)
  y = x + 1;
end
\end{verbatim}
\end{samepage}

This eachElem function puts 10 tasks into the sleigh workspace.  Each
task contains one value from 1 to 10.  This value is passed as the
argument to the add1 function.  The return value of the function is put
into the sleigh workspace.  The eachElem function waits for all of the
results to be put into the workspace, and returns them as a cell array,
which are numbers from 2 to 11.

As a second example, let's add two lists together.  We first define an
add2 function, which is written to a file called `add2.m':

\begin{samepage}
\begin{verbatim}
function z = add2(x, y)
  z = x + y;
end
\end{verbatim}
\end{samepage}

Then, we invoke the function with two variable arguments. Since we have two 
variable arguments, we have to wrap them in a cell array:

\begin{verbatim}
>> result = eachElem(s, 'add2', {1:5, 6:10});
\end{verbatim}

This is the parallel equivalent to the MATLAB command:

\begin{verbatim}
>> result = (1:5) + (6:10);
\end{verbatim}

We can keep adding more list arguments in this way, but there is also
a way to add arguments that are the same for every task, which we call
fixed arguments:

\begin{verbatim}
>> result = eachElem(s, 'add2', 1:10, 1);
\end{verbatim}

This is equivalent to the MATLAB command:

\begin{verbatim}
>> result = (1:10) + 1;
\end{verbatim}

It is also equivalent to invoking add1 function we previous defined.
\begin{verbatim}
>> result = eachElem(s, 'add1', 1:10)
\end{verbatim}

The order of the arguments passed to the function are normally in the
specified order, which means that the fixed arguments always come after
the varying arguments.  To change this order, a permutation vector can
be specified.  The permutation vector is specified using the
``argPermute'' field in the execOptions parameter.

For example, to perform the parallel equivalent of the MATLAB operation
100 - (1:10), we do the following:

\begin{samepage}
\begin{verbatim}
>> opt.argPermute = [2 1];
>> result = eachElem(s, 'sub2', 1:10, 100, opt);
\end{verbatim}
\end{samepage}

This permutation vector says to first use the second argument, and then
use the first, thus reverse the order of two arguments.

The sub2.m used in this example looks like this:

\begin{samepage}
\begin{verbatim}
function z = sub2(x, y) 
  z = x - y;
end
\end{verbatim}
\end{samepage}

There is another keyword argument, called ``blocking'', which, if set to
0, will make eachElem return immediately after submitting the tasks,
thus making it non-blocking.  A ``pending'' object is returned, which can
be used periodically to check how many of the tasks are completed, and also
to wait until all tasks are finished.  Here's a quick example:

\begin{samepage}
\begin{verbatim}
>> opt.blocking = 0;
>> sp = eachElem(s, 'add2', {1:1000, 1001:2000}, {}, opt)
>> while check(sp) > 0
     % Do something useful for a little while
   end
>> result = wait(sp)
\end{verbatim}
\end{samepage}

There is also a keyword argument called ``loadFactor'' that enables
watermarking.  This limits the number of tasks that are put into the
workspace at the same time.  This could be important if you're executing
a lot of tasks.  Setting the load factor to 3 limits the number of tasks
in the workspace to 3 times the number of workers in the sleigh workspace.  
Here's how to do it:

\begin{samepage}
\begin{verbatim}
>> opt.loadFactor = 3;
>> result = eachElem(s, 'add2', {1:100, 101:200}, {}, opt);
\end{verbatim}
\end{samepage}

The results are exactly the same as not using a load factor.  Setting
this option only changes the way that tasks are submitted by the
eachElem function.

As you can see, Sleigh makes it very easy to write simple parallel
programs.  But you're not limited to simple programs. You can use NWS
operations to communicate between the worker processes, allowing you to
write message passing parallel programs much more easily than using MPI
or PVM, for example.
