NetWorkSpaces (NWS) provides a framework to coordinate programs written
in scripting languages; NWS currently supports the languages Python,
MATLAB, and R.  This User Guide is for the MATLAB system, and it
assumes that the reader is familiar with MATLAB.

\section{Data Sharing}
A MATLAB program uses variables to communicate data from one part of the
program to another.  For example, \texttt{x = 123} assigns the value 123
to the variable named \texttt{x}.  Later portions of the program can
reference \texttt{x} to use the value 123.  This mechanism is generally
known as \textit{binding}.  In this case, the binding associates the
value 123 with the name \texttt{x}.  The collection of name-value
bindings in use by a program is often referred to as its
\textit{workspace}.

Two or more MATLAB programs use NWS to communicate data by means of
name-value bindings stored in a network-based workspace (a netWorkSpace,
which in MATLAB is an instance of a netWorkSpace object).  One program
creates a binding, while another program reads the value the binding
associated with the name. This is clearly quite similar to a traditional
workspace; however, a netWorkSpace differs from a traditional workspace
in two important ways.

First, in a setting in which two or more MATLAB programs are
interacting, it would not be unusual for one to attempt to ``read'' the
value of a name before that name has been bound to a value.  Rather than
receiving an ``unbound variable'' error, the reading program (by default)
simply waits (or ``blocks'') until the binding occurs. Second, a common
usage paradigm involves processing a sequence of values for a given
name. One MATLAB program carries out a computation based on the first
value, while another might carry out a computation on the second, and so
on. To facilitate this paradigm, more than one value may be bound to a
name in a workspace and values may be ``removed'' (fetch) as opposed to
read (find). By default, values bound to a name are consumed in
first-in-first-out (FIFO) order, but other modes are supported:
last-in-first-out (LIFO), multiset (no ordering implied) and single
(only the last value bound is retained). Since all its values could be
removed, a name can, in fact, have no values associated with it.

A netWorkSpace provides five basic operations:
\textbf{store}, \textbf{fetch}, \textbf{fetchTry}, \textbf{find}, and
\textbf{findTry}:
\begin{description}
\item[store] introduces a new binding for a specific name in a given workspace.
\item[fetch] fetches (removes) a value associated with a name, blocking
if necessary until a value is available.
\item[fetchTry] fetches (removes) a value associated with a name without
blocking.
\item[find] reads a value without removing it, blocking if necessary.
\item[findTry] reads a value without removing it and without blocking.
\end{description}

Note that \textbf{fetchTry} and \textbf{findTry} return \texttt{0} or
a user-supplied default if no value is available.

There are several additional netWorkSpace operations:
\begin{description}
\item[currentWs] returns the name of the specified workspace.
\item[declare] declares a variable name with a specific mode.
\item[deleteVar] deletes a name from a workspace.
\item[listVars] provides a list of variables (bindings) in a workspace.
\end{description}

In addition to a netWorkSpace, a MATLAB client of NWS also uses an
nwsServer object. This object is created automatically when a new
netWorkSpace object is created, so you don't need to interact directly
with it. However, the \texttt{server} method provides access to the
nwsServer object:

\begin{verbatim}
>> ws = netWorkSpace('foo')
>> server(ws)
\end{verbatim}

A nwsServer object supports the following actions:

\begin{description}
\item[openWs] connects to a workspace or creates one if the specified
workspace does not exist. 
\item[useWs] uses a netWorkSpace without claiming ownership.
\item[deleteWs] explicitly deletes a workspace.
\item[listWss] provides a list of workspaces in the server.  
\item[close] closes the connection to a workspace. Depending on the
ownership, closing a connection to a workspace can result in removing
the workspace. 
\end{description}

\section{Parallel Computing}
The operations above enable coordination of different programs using
NWS. There is also a mechanism, built on top of NWS, called Sleigh
(inspired by R's SNOW package) to enable parallel function evaluation.
Sleigh is especially useful for running {\it embarrassingly parallel}
programs on multiple networked computers. Once a sleigh is created by
specifying the nodes that are participating in computations, you can
use:

\begin{description}
\item[eachElem] is used to execute a specified function multiple times
in parallel with a varying set of arguments.
\item[eachWorker] is used to execute a function exactly once on every
worker in the sleigh with a fixed set of arguments.
\end{description}

Various data structures (workspaces, name-value bindings, etc.) in a NWS
server can be monitored and even modified using a web interface. In
distributed programming, even among cooperating programs, the state of
shared data can be hard to understand. The web interface presents a
simple way of checking current state remotely. This tool can also be
used for learning and debugging purposes, and for monitoring ``normal''
MATLAB programs as well.
