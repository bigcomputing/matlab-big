\section {nwsServer Class}
%HEVEA\aname{nwsServer}{
\rule[0.1in]{6in}{0.01in}
\newline
nwsServer
\newline
\rule{6in}{0.01in}
\begin{list}{}{}
	\item {\bf Description}\\
	nwsServer class constructor.
	\item {\bf Usage}\\
	nwss = nwsServer(h, p)
	\item {\bf Arguments}
		\begin{tabbing}
		h \hspace{2.5cm} \= server host name. Default is `localhost'.\\
		p \> server port number. Default is 8765.
		\end{tabbing}
	\item {\bf Examples}
	\begin{verbatim}
	example 1: create a nws server with default options.
	>> nwss = nwsServer;   % create nws server with default options
	
	example 2: create a nws server on node1, port 5000.
	>> nwss = nwsServer(`node1', 5000);
	\end{verbatim}
\end{list}
%HEVEA }

%HEVEA\aname{openWs}{
\rule[0.06in]{6in}{0.01in}
\newline
openWs
\newline
\rule{6in}{0.01in}
\begin{list}{}{}
	\item {\bf Description}\\
	Create a netWorkSpace class object. 
	\item {\bf Usage}\\
	nws = openWs(nwss, netWorkSpaceName, opts);
	\item {\bf Arguments}
		\begin{tabbing}
		nwss \hspace{2.5cm} \= a nwsServer class object. \\
		netWorkSpaceName \> name of the netWorkSpace to open.\\
		opts \> a structure of optional fields.
		\end{tabbing}
	\item {\bf Details}\\
	openWs(nwss, netWorkSpaceName) creates a shared netWorkSpace object for
	the netWorkSpace named netWorkSpaceName (which must be a string). 
	If the shared netWorkSpace does not exist, it will be created (upon first use)
	and claimed the ownership of the workspace. This can be useful as an arbitration
	mechanism. The output of listWss indicates which netWorkSpace this MATLAB process
	created. These will be destroyed when this MATLAB process closes the connection
	to the netWorkSpace server that was used to create these spaces, except when 
        when they are persistent workspaces. \\

	opts argument has two optional fields: 
	\begin{itemize}
	\item \textit {persistent}: a boolean value indicating whether the workspace is persistent or not. If
	the workspace is persistent (value of 1), then the workspace is not purged
	when the owner disconnects from the server. Default value is 0.
	\item \textit {space}: a netWorkSpace class object. 
	If this field is specified, then the existing netWorkSpace and server connection are used. Otherwise, a new netWorkSpace 
    class object with the connection to the nws server, nwss, is created.
	\end{itemize}
	
	\item {\bf Examples}
	\begin{verbatim}
	example 1: open a workspace with default options.
	>> nwss = nwsServer;
	>> nws = openWs(nwss, `my space');
	
	example 2: open a persistent workspace.
	>> nws2 = openWs(nwss, `my space2', opt);
	\end{verbatim}
\end{list}
%HEVEA }

%HEVEA\aname{useWs}{
\rule[0.06in]{6in}{0.01in}
\newline
useWs
\newline
\rule{6in}{0.01in}
\begin{list}{}{}
	\item {\bf Description}\\
	Connect to a shared netWorkSpace, but not own it. 
	\item {\bf Usage}\\
	nws = useWs(nwss, netWorkSpaceName, opts);
	\item {\bf Arguments}
		\begin{tabbing}
		nwss \hspace{3.0cm} \= a nwsServer class object. \\
		netWorkSpaceName \> name of the netWorkSpace to open.\\
		opts \> a structure of optional fields.
		\end{tabbing}
	\item {\bf Details}\\
	useWs(nwss, netWorkSpaceName, opts) creates a shared netWorkSpace object
	for the netWorkSpace named netWorkSpaceName (which must be a string).
	If the shared netWorkSpace does not exist, it will be created but no owner
	associates with it. \\
	
	opts argument contains the following field:
	\begin{itemize}
	\item \textit{space}: a netWorkSpace object that represents the workspace that client wants to `use', but not own. 
	If this field is specified, then the existing netWorkSpace and server connection are used. Otherwise, a new netWorkSpace 
    class object with the connection to the nws server, nwss, is created.
	\end{itemize}
		
	\item {\bf Examples}
	\begin{verbatim}
	>> nwss = nwsServer;
	>> nws = useWs(nwss, `my space');
	\end{verbatim}
\end{list}
%HEVEA }

%HEVEA\aname{listWss}{
\rule[0.06in]{6in}{0.01in}
\newline
listWss
\newline
\rule{6in}{0.01in}
\begin{list}{}{}
	\item {\bf Description}\\
	List all netWorkSpaces in the nwsServer, nwss.
	\item {\bf Usage}\\
	X = listWss(nwss);
	\item {\bf Arguments}
		\begin{tabbing}
		nwss \hspace{2.5cm} \= a nwsServer class object.
		\end{tabbing}
	\item {\bf Details}\\
	X = listWss(nwss) returns a struct array listing information
	about all the netWorkSpaces currently available on the server reached by connection nwss.

	If no left hand side, print a tabular listing. 
		
	\item {\bf Examples}
	\begin{verbatim}
	>> nws = netWorkSpace(`my space');
	>> store(nws, `a', 1);
	>> store(nws, `b', 2);
	>> nwss = server(nws);
	>> listWss(nwss);
	\end{verbatim}
\end{list}
%HEVEA }

%HEVEA\aname{mktempWs}{
\rule[0.06in]{6in}{0.01in}
\newline
mktempWs
\newline
\rule{6in}{0.01in}
\begin{list}{}{}
	\item {\bf Description}\\
	Create a unique temporary workspace using template string.

	\item {\bf Usage}\\
	name = mktempWs(nwss, wsNameTemplate);

	\item {\bf Arguments}
		\begin{tabbing}
		nwss \hspace{2.5cm} \= a nwsServer class object. \\
		wsNameTemplate \> template for the netWorkSpace name.
		\end{tabbing}
	\item {\bf Details}\\
	name = mktempWs(nwss, wsNameTemplate) returns the name of a temporary space created
	on the server reached by nwss. The template should contain a `\%d'-like construct
	which will be replaced by a serial counter maintained by the server to generate
	a unique new workspace name. The user must then invoke openWs() or useWs() with this name
	to create a netWorkSpace object to access this workspace. wsNameTemplate defaults to `matlab\_ws\_\%010d'.
		
	\item {\bf Examples}
	\begin{verbatim}
	>> nwss = nwsServer;
	>> nws = mktempWs(nwss, `template_%010d');
	\end{verbatim}
\end{list}
%HEVEA }

%HEVEA\aname{socket}{
\rule[0.06in]{6in}{0.01in}
\newline
socket
\newline
\rule{6in}{0.01in}
\begin{list}{}{}
	\item {\bf Description}\\
	returns socket connection of a nwsServer, nwss.	
	\item {\bf Usage}\\
	s = socket(nwss);
	\item {\bf Arguments}
		\begin{tabbing}
		nwss \hspace{2.5cm} \= a nwsServer class object.
		\end{tabbing}	
	\item {\bf Examples}
	\begin{verbatim}
	>> nwss = nwsServer;
	>> socket(nwss);
	\end{verbatim}
\end{list}
%HEVEA }

%HEVEA\aname{deleteWs}{
\rule[0.06in]{6in}{0.01in}
\newline
deleteWs
\newline
\rule{6in}{0.01in}
\begin{list}{}{}
	\item {\bf Description}\\
	Delete a shared netWorkSpace with specific name from the nwsServer, nwss. 
	\item {\bf Usage}\\
	X = deleteWs(nwss, wsName);
	\item {\bf Arguments}
		\begin{tabbing}
		nwss \hspace{2.5cm} \= a nwsServer class object. \\
		wsName \> name of netWorkSpace.
		\end{tabbing}	
	\item {\bf Examples}
	\begin{verbatim}
	>> nwss = nwsServer;
	>> deleteWs(nwss, `my space');
	\end{verbatim}
\end{list}
%HEVEA }
